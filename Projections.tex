
 \documentclass[12pt,leqno]{amsart}
\usepackage[dvips]{graphics}
\usepackage{amssymb,hyperref}
\usepackage{color,comment}
\newenvironment{bew}{\begin{proof}[Beweis]}{\end{proof}}
%\usepackage{epsfig}
%\typein[\answer]{To run syntax check, enter `S';
% otherwise, just press return:}
% \let\SANSWER\syntaxonly     \let\sANSWER\syntaxonly
% \csname\answer ANSWER\endcsname
%\newenvironment{bew}{\proof[\proofname]}{\endproof}

\numberwithin{equation}{section}

%\voffset=-.7in 
%\hoffset=-.7in
%\setlength{\textwidth}{7in}

%\setlength{\textheight}{8.5in}

\newtheorem{thm}{Theorem}[section]
\newtheorem{thmmain}{\tref{thm-main}}[section]
\renewcommand{\thethmmain}{}
\newtheorem{lem}[thm]{Lemma}
%\newtheorem{defn}[thm]{DEFINITION}
\newtheorem{cor}[thm]{Corollary}
\newtheorem{prop}[thm]{Proposition}
\newtheorem{conj}[thm]{CONJECTURE}
\newtheorem{quest}[thm]{Problem}
\newtheorem{fak}[thm]{Fakt}
\newtheorem{kor}[thm]{Korollar}
\newtheorem{hyp}[thm]{HYPOTHESES}
\newtheorem{ft}[thm]{FAKT} 
\newtheorem{satz}[thm]{SATZ}
\newtheorem{defn}[thm]{Definition}
%\newtheorem{proof}[pf]{Beweis}
\theoremstyle{remark}
%\newtheorem{rem}{Remark}[section]
\newtheorem{rem}[thm]{Remark}
\newtheorem{ex}[thm]{Example}

\newcommand{\tref}[1]{Theorem~\ref{#1}}
\newcommand{\cref}[1]{Korollar~\ref{#1}}
\newcommand{\pref}[1]{Proposition~\ref{#1}}
\newcommand{\rref}[1]{Remark~\ref{#1}}
\newcommand{\dref}[1]{Definition~\ref{#1}}
\newcommand{\secref}[1]{\S\ref{#1}}
\newcommand{\lref}[1]{Lemma~\ref{#1}}
\newcommand{\exref}[1]{Example~\ref{#1}}
\newcommand{\sref}[1]{Satz~\ref{#1}}
\newcommand{\fref}[1]{Fakt~\ref{#1}}
\newcommand{\Ant}{\mathrm{Ant}} 
\newcommand{\diam}{\mathrm{diam}}
\newcommand{\Pol}{\mathrm{Pol}}
%\newcommand{\vol}{\mathrm{vol}}
\newcommand{\pt}{\mathrm{pt}}
\newcommand{\rad}{\mathrm{rad}}
\newcommand{\Rad}{\mathrm{Rad}}
\newcommand{\Sub}{\mathrm{Sub}}
\newcommand{\Iso}{\mathrm{Iso}}
\newcommand{\Rm}{\mathrm{Rm}}
\newcommand{\Rg}{\mathrm{Rg}}
\newcommand{\Fix}{\mathrm{Fix}}
\newcommand{\R}{\mathbb{R}}
\newcommand{\C}{\mathbb{C}}
\newcommand{\grad}{\mathrm{grad}}

\newcommand{\curv}{\operatorname{curv}}
\DeclareMathOperator\Hess{Hess}
\DeclareMathOperator\vol{vol}
\newcommand{\eps}{\varepsilon}
\def\X{{X^\varepsilon}}
\def\di{{\bold div}\,}

\newcommand{\Reg}{\mathrm{Reg}}

\newcommand{\idm}{\mathrm{Id}}

\newcommand{\green}{\color{green}}
\newcommand{\red}{\color{red}}
\newcommand{\cyan}{\color{cyan}}
\newcommand{\blue}{\color{blue}}
\newcommand{\nc}{\normalcolor}
\def\co{\colon\thinspace}
\def\uk{\bar K}
\def\lk{\underline K}
\def\uk{\bar K}

\def\small{\theta}
\newcommand{\Ric}{\mathrm{Ric}}

\newcommand{\snk}[1]{\sin_{#1}}
\newcommand{\csk}[1]{\cos_{#1}}
\newcommand{\ctk}[1]{\cot_{#1}}
\newcommand{\mdk}[1]{\mathrm{md}_{#1}}
\newcommand{\Idm}{\mathrm{Id}}
%\newcommand{\curv}{\mathrm{curv}}

\pagestyle{plain}

\begin{document}
	%\tableofcontents
	\pagebreak
	%\bibliographystyle{alpha}
	
	%\pagenumbering{roman}
	
	\title{Alternating projections in Hadamard spaces}
	
	
	\thanks{V.K.  is partially supported by a Discovery grant from NSERC;
		A. L. was partially supported by the DFG grants   SFB TRR 191 and SPP 2026.}

	
	\author{Alexander Lytchak and Anton Petrunin}
	
%	\address
%	{Mathematisches Institut\\ Universit\"at K\"oln\\ Weyertal 86 -- 90\\ 50931 K\"oln, Germany}
%	\email{alytchak@math.uni-koeln.de}
	
	\keywords
	{CAT(0), metric porjections, convex subset}
	\subjclass
	[2010]{53C20, 53C21, 53C23}
	
	
	\date{\today}
	
	%\thanks{The second author was partially supported by Swiss National Science Foundation Grant 153599}
	
	
	\begin{abstract}
		We  prove that  alternating projections onto convex subsets of Hadamard spaces can behave in a more complicated way than in Hilbert spaces. We show that the asympotic regularity theorem of Bauschke does not need to hold in Hadamard spaces resolving a problem formulated by Miroslav Bacak. 
	\end{abstract}
	
	
	
	
	%\author{Alexander Lytchak}
	%\address{Mathematisches Institut\\ Universit\"at Bonn\\
	%Wegelerstrasse 10, 53115 Bonn, Germany\\}
	%\email{lytchak\@@math.uni-bonn.de}
	
	
	%\subjclass{53C20,  52B99}
	%\footnotetext[1]{}
	
	
	\maketitle


%\author{Alexander Lytchak}
%\address{Mathematisches Institut\\ Universit\"at Bonn\\
%Wegelerstrasse 10, 53115 Bonn, Germany\\}
%\email{lytchak\@@math.uni-bonn.de}


%\subjclass{53C20,  52B99}
%\footnotetext[1]{}






%\section{Asymptotic regularity of compositions of projections}
\section{Introduction}
%In this paper we show that alternating projections on convex subsets can behave in a Hadamard space in a much more complicated  way than in a Hilbert space.





Let  $C_1,..., C_n $ be closed convex subsets in a Hilbert space $H$ and denote by  $P_{C_i}$ the  metric projection  $P_{C^i} :H\to C_i$ sending point $x\in H$ to the point in $C_i$ closest to $x$.  The properties of the composition 
$$P= P_{C_1}\circ P_{C_2}\circ .... \circ P_{C_n} :H\to C_1$$
and its iterates $P^m :C_1 \to C_1$ is a classical object of investigation
 in optimization theory, see \cite{}, \cite{} and the references therein for a huge bibliography on this subject.


 % The composition $P$ and its iterates are contracting (i.e. $1$-Lipschitz) and basic questions concerns the existence of fixed points and properties  of sequences
 % $x_m=P^m (x)$ for an arbitrary starting point $x$.  




More recently  compositions of projections were investigated in  Hadamard spaces, thus in complete, geodesic metric spaces which are globally non-positively curved.    In this situation the metric projection to any closed convex subset is also well-defined and contracting (i.e. 1-Lipschitz).    For a finite sequence $C_1,...,C_n$ of convex subsets having a common intersection point $z$,   the composition $P=  P_{C_1}\circ P_{C_2}\circ .... \circ P_{C_n} :H\to C_1$ behaves similarly as in the Hilbert space case, see \cite{asymptotic}, \cite{Bac2}, \cite{Bac}.
  

If the convex sets $C_1,...,C_n$ of a Hilbert space have empty intersection, the fundamental result of Heinz Bauschke, \cite{Bauschke}  (see also \cite{Kohlenbach}),   
 states that, for any $x\in H$, the sequence $x_m=P^m (x)$  is \emph{asymptotically regular},
 meaning that the distances $d(x_m,x_{m+1})$ converge to $0$ for $m\to \infty$.
It has been speculated if   also in Hadamard spaces alternating projections result in asymptotically regular sequences \cite[Problem 6.13]{Bac}.
In this paper we show that alternating projections on convex subsets can behave in a Hadamard space in a much more complicated  way than in a Hilbert space:


\begin{thm}\label{thm}
There exists a compact Hadamard space $X$ and $3$ convex subsets $C_1,C_2,C_3$ in $X$, each of them isometric to the interval 
$[-1,1]$,  such that the composition $P= P_{C_1}\circ P_{C_2} \circ P_{C_3} :C_1\to C_1$ is the non-trivial isometry of $C_1$ interchanging the endpoints $x_+,x_-$ of the interval $C_1$. In particular, the sequence $P^m (x_+)$ is  alternating between $x_+$ and $x_-$, hence it is not asymptotically regular.
\end{thm}



The proof is rather simple once you believe in the statement of the theorem. We just need to construct a CAT(0) space which contains 3 totally convex  flat quadrangles $Q_1,Q_2,Q_3$, such that $Q_i$ and $Q_{i+1}$ share exactly one side (the segment $C_i$) and such that the union of the quadrangles  is homeomorphic to a Moebius-strip.   


  It is not very difficult to vary the construction and obtain a compact Hadamard space $X$ and a finite sequence of convex subsets $C_1,...,C_n$ all isometric to the unit disc, such that the composition $P=P_{C_1}\circ ...\circ      P_{C_n} :C_1\to C_1$  is a rotation of the disc by an arbitrary (especially an irrational) angle. 



On the other hand, is is  not difficult to show, that in case of two subsets the result of \cite{Bauschke} generalizes:

\begin{prop} \label{prop}
For any closed convex subsets $C_1,C_2$ of any Hadamard space $X$, for any starting point $x\in X$  the sequence $x_m = P^m (x)$ with $P=  P_{C_1}\circ P_{C_2} :X\to C_1$ is asymptotically 
regular.
\end{prop}

The proof shows   $d(x_m,x_{m+1}) \leq \frac  {A} {\sqrt m}$, for some $A>0$.






\section{Two subsets}
We will assume some familiarity with the geometry of CAT(0) spaces, \cite{BH}, \cite{AKP}.
%\begin{thm}
%Let $X$ be a CAT(0) space, let $C_1,C_2$ be closed convex subsets, let $P_i:X\to C_i$ denote the nearest-point projection and let 
%$P$ be the composition $P_1\circ P_2$.   Then for any point $x\in X$, the sequence $x_n= P^n (x)$ is asymptotically regular,
%hence it satisfies $\lim _{n\to \infty} d(x_x,x_{n+1})=0$.
%\end{thm}
We are going to prove Proposition \ref{prop} in this Section.



\begin{proof}[Proof of Proposition \ref{prop}]
By definition,  $x_m \in C_1$ for all $m \geq 1$. Let  $y_m= P_2 (x_m)$ denote the metric projection of $x_m$ to $C_2$.  
Set 
$$r_m:=d(x_m,x_{m+1}) \; ; \; s_m: = d(y_m,y_{m+1}) \;,$$
and note that $r_m \geq s_m \geq r_{m+1}$ for all $m$, by the definition of metric projection.
We further set
 $$  a_m: = d(x_m,y_m)= d(x_m,C_2) \; ; \;
 b_n:=d(y_m, x_{m+1})= d(y_m,C_1) \;$$
and note that  $a_m \geq b_m \geq a_{m+1}$.

The CAT(0) assumption implies  $b_m^2+r_m^2  \leq a_m^2$. Therefore,  
$$b_m \leq a_m - \frac {r_m^2} {2 a_m}   \leq a_m - \frac {r_m^2 } {2\cdot a_1} \;.  $$ 
This implies $\lim _{m\to \infty} r_m ^2 < \infty$.  Since the sequence $r_m$ is non-increasing, we deduce $r_m \leq A \cdot m ^{-\frac 1 2 }$, for some $A>0$. 
\end{proof}



\section{Main example}
In this  Section we construct an example proving Theorem \ref{thm}.


\begin{proof}[Proof of Theorem \ref{thm}]
Fix a sufficiently small number $\epsilon >0$.  Consider the round sphere $S^2$ and the closed 
$\epsilon$-neighborhood $A$  of its equator.   The space $A$ is locally CAT(1) (as one can see, for instance, by applying  \cite{LWcurv}) and the universal covering $\tilde A$ of $A$ with its induced metric is  CAT(1), \cite{Ballmann}, since $\tilde A$ has no closed geodesics.

Denote by $E$ the preimage of the equator in $\tilde A$  and by $H^{\pm}$ the boundary curves of the stripe $\tilde A$, which are both convex in $\tilde A$.    The isometry group of $\tilde A$ 
contains the group of translations  along $E$ (the preimage of rotations of the sphere) and 
the reflection at $E$, interchanging $H^+$ and $H^-$.  Let $g$   be the composition of translation along $E$  of length $2\pi +\delta$ (for some small $\delta >0$) and the reflection $r$.  The element $g$ generates a discrete group $\Gamma$ isomorphic to $\mathbb Z$ in the group of isometries  of $\tilde A$. The group $\Gamma$ acts freely on $\tilde A$ and the quotient space $Y =\tilde A/\Gamma$ is homeomorphic to the Moebius-band.

Once $\delta>0$ is fixed, we can choose a sufficiently small $\epsilon >0$, such that every point of $\tilde A$ is moved by any element of $\Gamma$ more than $2\pi$ away from it.  This implies that the locally CAT(1) space $Y$ does not  contain geodesics of length less than $2\pi$, hence $Y$ is a CAT(1) space.  By construction, $Y$ is locally isometric to $S^2$ outside the boundary curve $B$, which is the projection of $H^+$  (and of $H^-$).  The projection of $E$ to $Y$ is a closed geodesic $G$ of length $2\pi +\delta$.



Denote by $X$ the Euclidean cone over $Y$.   By construction and \cite{BBI}, the space $X$ is CAT(0), moreover, $X$ is locally Euclidean outside the "boundary", the cone over $B$.

  The cone $Z$  over the  geodesic $G$ is a locally covex, hence a convex subset of $X$.
In the space $Z$  (just a cone over a circle of length $2\pi +\delta$ )   we consider a triangle
$q_1,q_2,q_3$ 
in the topological plane $Z$, which contains the origin $o$ of the cone in its interior.


By construction, the triangle $q_1q_2q_3 \subset Z\subset X$ is contained in the flat part of $X$.
Thus, we fine a small neighborhood $U_1$ of the geodesic $q_1q_2$ isometric to a convex subset of the Euclidean space.  Similarly, we find neighborhood $U_2$ pf $q_2q_3$ and $U_3$ of $q_3q_1$.
We find some $r>0$, such that the $r$-neighborhood of $q_iq_{i+1}$ is completely contained in $U_i$. 

 Denote by $C_i$ the segment centered at $q_i$ and being orthogonal to $Z$ at $q_i$.
By construction, $C_i$ and $C_{i+1}$, for $i=1,2,3$   are contained in $U_i$.  Since $Z$ is convex, $C_i$ and $C_{i+1}$ are "parallel" inside $U_i$, thus their convex hull  $Q_i$ is isometric to a rectangle with two parallel sides given by $C_i$ and $C_{i+1}$.  In particular, the metric projection from $C_{i+1}$ to $C_{i}$ is an isometry.   

By topological reasons, the union of $Q_1,Q_2,Q_3$ is homeomorphic to the Moebius strip and not to the annulus. Thus, going "around" we change the orientation of $C_1$. This just means, that the composition $P:C_1\to C_1$, which we is an isometry, changes the orientation of the segment $C_1$.	
	\end{proof}






\bibliographystyle{alpha}
\bibliography{Projection}










\end{document}	


