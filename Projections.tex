\documentclass[a4paper,10pt]{article}
\usepackage{paper}

\def\thetitle{Cyclic projections in Hadamard spaces}
\def\theauthors{Alexander Lytchak and Anton Petrunin}
\hypersetup{
pdftitle={\thetitle},
pdfauthor={\theauthors}
}

\begin{document}


\title{\thetitle}	
\author{\theauthors}
\date{}
\maketitle

\begin{abstract}
We  prove that iterating projections onto convex subsets of Hadamard spaces can behave in a more complicated way than in Hilbert spaces, resolving a problem formulated by Miroslav Bačák.
\end{abstract}

%\begin{keywords}
%CAT(0), Hadamard space, asymptotic regularity, metric projection
%\end{keywords}
%\subjclass[2010]{53C20, 53C21, 53C23}
%\pagestyle{empty}\thispagestyle{empty}

\section{Introduction}

The method of cyclic  projections is a classical algorithm seeking  an intersection point of a finite family  $C_1,\dots, C_k$ of  closed convex subsets in a \emph{Hilbert  space}~$X$.
Denote by $P_i$ the closest-point projection $X\to C_i$; it sends a point $x\in X$ to the (necessarily unique) point $P(x)$ in $C_i$ that minimize the distance to~$x$.
Given a point $x\in X$ consider the sequence $x_n=P^n(x)$, where
$P$ is the  cyclic composition of projections $P= P_1\circ \dots \circ P_k$.  The method of cyclic projections analyzes the sequence $(x_n)$, tries to find a limit point $x_{\infty}$, to show $x_{\infty} \in C_1\cap\z\dots\cap C_k$ and to understand the  rate of convergence.

We  cite the central results concerning this method and refer to [9],

 F. Deutsch, “The method of alternating orthogonal projections,” in Ap-
proximation theory, spline functions and applications. Springer, 1992,
pp. 105–121. 


and \cite{Bac, Bac2} for the history, applications, generalizations  and recent developments. If the intersection  $C_1\cap ...\cap C_k$ is non-empty, then $(x_n)$ always converges weakly to a limiting point $x_{\infty} \in  C_1\cap ...\cap C_k$,

Bregman,  The method of successive projection for finding a common point of convex sets,” Sov. Math. Dok., vol. 162, no. 3, pp. 688–692,
1965



 however, this convergence does not need to be strong
 
 H. S. Hundal, An alternating projection that does not converge in
 norm, Nonlinear Analysis: Theory, Methods \& Applications, vol. 57,
 no. 1, pp. 35–61, 2004.
 
 
   If, in addition,  $C_i$ are  linear subspaces, then the convergence is strong

 J. Von Neumann, “On rings of operators. Reduction theory,” Annals of
Mathematics, pp. 401–485, 1949),15] 

I. Halperin, “The product of projection operators,” Acta Sci.
Math.(Szeged), vol. 23, no. 1, pp. 96–99, 1962).


 If the intersection  $C_1\cap ...\cap C_k$ is not assumed to be non-empty,  the analysis of the sequence $(x_n)$ is more complicated. However, in [8] it has been established 
 that the cyclic product  $P= P_1\circ \dots \circ P_k$ is \emph{asymptotically regular};
by definition this means, that for any starting point  $x\in X$, we have  $|x_n-x_{n+1}|\to 0$ as $n\to \infty$.

The rates of convergence, respectively, the rates of asymptotic regularity have been investigated in several works, see, for instance [9] and [14].



 More recently, the method of cyclic  projections has been investigated beyond the setting of Hilbert spaces in so called Hadamard spaces (also known as CAT(0) spaces, or globally non-positively curved spaces). This class of metric spaces includes hyperbolic spaces, metric trees, Riemannian manifolds of negative non-positive curvature; it has played an important role in many areas of mathematics  in the last decades.  We assume some familiarity with Hadamard spaces, refer the reader to \cite{BBI,BH,AKP,AKP_inv} as general references on this subject and to 
 
 [Bacak, Convex Analysis and Optimization in Hadamard spaces] 
 
 and
 [5, Section 6.8], and the bibliography therein,  for the introduction and applications of the method of alternating projection in Hadamard spaces.

Hadamard spaces are defined (loosely speaking) by the property that their distance function is at least as convex as the distance function on a Hilbert space. In particular,
Hadamard spaces contain a huge variety of convex subset; closest point projections to closed convex subsets are well-defined and $1$-Lipschitz and the questions discussed above about cyclic products $P= P_1\circ \dots \circ P_k$ of closest-point projections $P_i:X\to C_i$ onto closed convex subsets $C_i$ of a Hadamard space $X$ are absolutely meaningful.

Many results discussed above have been transferred from the linear setting of Hilbert space
to general Hadamard spaces.  If the subsets $C_i$ have a non-empty intersection then 
the cyclic projections $p_n=P^n (p)$ converge \emph{weakly} to a point $p_{\infty} \in C_1\cap ...\cap C_k$, [6], [4]; see [5], 

[M. Bacak. Convex analysis and optimization in Hadamard spaces. De
Gruyter, Berlin, 2014]

[Our paper {Weak topology}] ,  

for the property of weak topology on Hadamard spaces.   Moreover, if the convex subsets $C_i$ have a non-empty intersection, then the cyclic projection $P$ is asymptotically regular.  We refer to 


Kohlenbach, Ulrich (D-DARM); López-Acedo, Genaro (E-SEVL-MA); Nicolae, Adriana (R-CLUJ)
Quantitative asymptotic regularity results for the composition of two mappings. (English summary)
Optimization 66 (2017), no. 8, 1291–1299. 

for the investigation of the rate of convergence in this setting.

Somewhat surprisingly, it turns out, that the fundamental result of Heinz Bauschke [8] for (possibly) non-intersecting convex subsets $C_i$ does not admit  a generalization to the
 setting of Hadamard spaces.  The following main result of this paper provides a negative answer to  the question of Miroslav Bačák \cite[Problem 6.13]{Bac}.

We will denote by $|x-y|$ the distance between points $x$ and $y$ in any metric space, even without linear structure.

\begin{thm}{Theorem}\label{thm}
There exists a Hadamard space $X$ and compact convex subsets $C_1,\dots,C_k$ in $X$ such that the composition of the closest-point projections $P\z= P_1\circ \dots \circ P_k$ is not asymptotically regular.
\end{thm}

%If the sets $C_1,\dots,C_k$ have a common intersection, then such examples are impossible \cite{asymptotic,Bac2,Bac}.

We provide an explicit example with $X$ being a product of two trees, proving the theorem  for  $k=3$.
Setting $C_3=\dots=C_k$ defines examples for any $k\geq 3$.  

In this example, all subsets $C_i$ is isometric to an interval of length $2$, the projections $P_i$ maps all of this segments isometrically onto $C_1$ and the composition $P=P_1\circ P_2\circ P_3$ is maps $C_1$ to itself isometrically but exachanges the endpoints.      While $P$ in this example is not asymptotically regular, the square $P^2$ is the identity on $C_1$, in particular, asymptotically regular. 


It is possible, to vary the construction  and to obtain an example for which no power of $P$ is asymptotically regular. This examples requires a somewhat deeper understandig of the geometry of Hadamard spaces.   We do not provide this example her and refer the interested reader to the preliminary arXiv version of our paper. 

%In the first example (Section \ref{sec:tripods}), the constructed space is a product of two tripods;
%it contains three convex flat quadrangles $Q_1$, $Q_2$, and $Q_3$ with pairs of opposite sides $(C_1,C_2)$, $(C_2,C_3)$, and $(C_3,C_1)$ such that the composition of projections swaps the ends of $C_1$.

%In the second example (Section \ref{sec:discs}), the convex subsets $C_1,C_2,C_3$ are isometric to the unit disc, and the composition $P\:C_1\z\to C_1$ rotates the disc by an arbitrary angle.
%If the angle is chosen irrational, then no power of $P$ is asymptotically regular.
 
 On the other hand, for two convex subsets, thus in the case $k=2$,  the result of Heinz Bauschke \cite{Bauschke} admits a generalization;  in this case the algorithm is simple enough to be controlled explicitly, even providing an optimal rate of asymptotic regularity.  

\begin{thm}{Proposition} \label{prop}
Let $C_1,C_2$ be two closed convex subsets of a Hadamard space $X$.
Then the composition $P\z=  P_1\circ P_2$ is asymptotically  regular.

Moreover, $|x_n-x_{n+1}| =o (\frac  {1} {\sqrt n})$ for any  $x\in X$ and $x_n \z= P^n (x)$.
\end{thm}

Examples given by the real axes $C_1 \subset \R^2$ and the set
\[C_2  = \{\,(x,y):x>0, y \geq 1+ x^{ -\epsilon}\,\}\]
reveal that the convergence rate cannot be improved to $O (n^{-\frac 1 2  -\epsilon})$ for any $\epsilon >0$. 

 This also shows, that the optimal rate of asympotitic regularity for cylcic projections on two convex subsets is the same for the Euclidean plane and for general Hadamard spaces. 

%\medskip

%Further, we assume familiarity with the geometry of Hadamard spaces \cite{BBI,BH,AKP,AKP_inv,ballmannbook}.

\parbf{Acknowledgments.}
We thank Miroslav Bačák and Nina Lebedeva for helpful comments and conversations.
Alexander Lytchak was partially supported by the DFG grant, no. 281071066, TRR 191.
Anton Petrunin was partially supported by the NSF grant, DMS-2005279.

\section{Three segments in a product of two tripods}\label{sec:tripods}

A union of three unit segments that share one endpoint with the induced length metric will be called a \emph{tripod}.  
Consider two tripods $S$ and $T$ and the product space $X= S\times T$. The space $X$ is a product of two trees, thus of two Hadamard spaces. Hence $X$ is a Hadamard space.



Denote by $a$, $b$, $c$ and $x$, $y$, $z$ the sides of $S$ and $T$ respectively.
\begin{figure}[h!]
\vskip0mm
\centering
\includegraphics{mppics/pic-20}
\end{figure}

%By the Reshetnyak gluing theorem, $S$ and $T$ are CAT(0).
%Therefore, so is the product space $X=S\times T$.
The following diagram shows 3 isometric copies of $2{\times}2$-square in $X$; they are obtained as the products of two pairs of sides in $S$ and $T$ as labeled:
\begin{figure}[ht!]
\vskip0mm
\centering
\includegraphics{mppics/pic-30}
\end{figure}

Consider the segment $C_1$, $C_2$, and $C_3$ shown on the diagram;
they all have slope $-1$ and project to each other isometrically.
Note that each projection $P_i$ reverses the shown orientation.
It follows that the cyclic projection $P$ sends the segment  $C_1$ to itself isometrically and changes the orientation of the segment. In particular, $P$ exchanges the ends of the segment, hence $P$ is not asymptotically regular: For $p$ being an end of the segment, and any $n$, we have 
 $|P^n (p)-P^{n+1}(p)|=2$.
  






%\section{Three discs}\label{sec:discs}

%Fix an angle $\alpha$ and a small $\epsilon>0$.
%Consider the closed 
%$\epsilon$-neighborhood $A$ of a closed geodesic in the unit sphere~$\mathbb{S}^3$.
%By the result of Stephanie Alexander,  David Berg, and Richard Bishop \cite {ABB-1993}, the space $A$ is locally CAT(1).
%The universal cover $\tilde A$ of $A$ with its induced metric is locally CAT(1) as well. 

%Denote by $E$ the preimage of the geodesic in $\tilde A$.
%The isometry group of $\tilde A$ 
%contains the group of translations along $E$ and 
%the rotations that fix $E$.
%Let $T$  be the composition of translation along $E$  of length $2\pi +10\cdot\epsilon$ and the rotation by angle $\alpha$.
%The element $T$ generates a discrete subgroup $\Gamma$ in the group of isometries  of $\tilde A$ that acts freely.

%Set $Y =\tilde A/\Gamma$.
%Since $\eps$ is small, every point of $\tilde A$ is moved by any element of $\Gamma$ more than $2\cdot\pi$.
%Therefore, $Y$ is a compact locally CAT(1) space that does not contain closed geodesics of length less than $2\pi$.
%Hence, by  the generalized Hadamard--Cartan theorem \cite{AKP}, $Y$ is CAT(1).
%By construction, $Y$ is locally isometric to $\mathbb{S}^3$ outside its boundary $B$.
%The projection of $E$ to $Y$ is a closed geodesic $G$ of length $2\pi +10\cdot\epsilon$.

%Denote by $X$ the Euclidean cone over $Y$;
%since $Y$ is CAT(1), we get that $X$ is CAT(0); see \cite{AKP}.
%Moreover, $X$ is locally Euclidean outside its \emph{boundary} --- the cone over $B$.

%The cone $Z$ over the closed geodesic $G$ is  the Euclidean cone over a circle of length $2\pi +10\cdot\epsilon$.
%By construction, $Z$ is locally convex, hence a convex subset of~$X$ \cite[2.2.12]{AKP_inv}.
%Let us consider a geodesic triangle
%$[q_1q_2q_3]$
%in $Z$ that surrounds the origin $o$ of the cone $Z$.

%By construction, the sides of triangle $[q_1q_2q_3]$ lie in the flat part of $X$.
%Thus, we can find a small $r>0$ such that the $2\cdot r$-neighborhood $U_1$ of the geodesic $[q_1q_2]$ isometric to a convex subset of the Euclidean space.
%We can assume that $2\cdot r$-neighborhoods $U_2$ of $[q_2q_3]$ and $U_3$ of $[q_3q_1]$ have the same property.

%Denote by $C_i$ the disc of radius $r$ centered at $q_i$ and being orthogonal to~$Z$.
%By construction, $C_i$ and $C_{i+1}$, for $i=1,2,3\pmod 3$   are contained in~$U_i$.
%Since $Z$ is convex, $C_i$ and $C_{i+1}$ are \emph{parallel} inside $U_i$, thus their convex hull  $Q_i$ is isometric to a right cylinder with bases $C_i$ and $C_{i+1}$.
%In particular, the projection $P_i$ defines an isometry $C_{i+1}\to C_{i}$.

%By construction, the composition $P=P_1\circ P_2\circ P_3\:C_1\to C_1$ rotates $C_1$ by angle $\alpha$.
%If $\tfrac\alpha\pi$ is irrational, then $P$, as well as all its powers, are \emph{not} asymptotically regular.


\section{Two sets}


\begin{wrapfigure}{r}{50mm}
\vskip-11mm
\centering
\includegraphics{mppics/pic-10}
\end{wrapfigure}

\mbox{\parit{Proof of \ref{prop}.}}
By definition,  $x_n \in C_1$ for all~$n$.
Set $y_n= P_2 (x_n)$, so $y_n$ is the closest-point projection of $x_n$ to $C_2$.
Further set 
\begin{align*}
r_n&:=|x_n-x_{n+1}|,\\
s_n&:=|y_n-y_{n+1}|.
\end{align*}
Since the closest-point projection is nonexpanding, we get
\[r_1 \geq s_1 \geq r_2\geq s_2\geq\dots
\eqlbl{eq:rsrsrsrs}
\]

Set
\begin{align*}
a_n &\df |x_n-y_n|= \dist_{C_2}x_n,\\
 b_n &\df |y_n-x_{n+1}|= \dist_{C_1}y_n.
\end{align*}
Note that
\[a_1 \geq b_1 \geq a_2 \geq b_2 \geq \dots
\eqlbl{eq:abababa}\]

Since $C_1$ is convex and $x_{n+1}\in C_1$ lies at the minimal distance from $y_n$, we have $\measuredangle[x_{n+1}\,{}^{x_n}_{y_n}]\ge \tfrac\pi2$. 
Since $X$ is a Hadamard space,
\[r_n^2  \leq a_n^2-b_n^2.\]
Therefore, \ref{eq:abababa} implies that 
\[\sum_{n} r_n ^2\le a_1^2.\]
By \ref{eq:rsrsrsrs}, $r_n$ is non-increasing.
Therefore, $r_n = o(\tfrac1{\sqrt{n}})$.
\qeds


{\sloppy
\printbibliography[heading=bibintoc]
\fussy
}

\end{document}	


